\documentclass{article}
\usepackage{subaru_queue}
\begin{document}
\semester{}
\proposalid{}
\receivedate{}

%%%%%%%%%%%%
%% page 1 %%
%%%%%%%%%%%%

%%%%%%%%%% 1. Title of Proposal %%%%%%%%%%
\title{Deep optical imaging of the Palomar 5 and GD-1 stellar streams with HSC}

%%%%%%%%%% 2. Principal Investigator's information %%%%%%%%%%
\PIfirstname {Adrian}
\PIlastname  {Price-Whelan}
\PIinstitute {Princeton University}
\PIaddress   {4 Ivy Lane, Peyton Hall, Princeton, NJ 08544, USA}
\PIemail     {adrn@astro.princeton.edu}
\PIphone     {}

%%%%%%%%%% 3. Scientific Category %%%%%%%%%%
% Uncomment ONE of the following lines to indicate
% the scientific category
%\SolarSystem
%\ExtrasolarPlanets
%\StarFormationandYoungDisk
%\ISM
%\NormalStars
%\MetalPoorStars
%\CompactObjectsandSNe
\MilkyWay
%\LocalGroup
%\NearbyGalaxies
%\IGMandAbsLineSystems
%\Cosmology
%\GravitationalLenses
%\ClustersandProtoClusters
%\GalaxyPropertiesandEnvironment
%\HighzGalaxiesLAEsLBGs
%\HighzGalaxiesothers
%\AGNandQSOActivity
%\Miscellaneous

%%%%%%%%%% 4. Abstract %%%%%%%%%%
\begin{abstract}
    % TODO: max 11 lines in PDF!
\end{abstract}

%%%%%%%%%% 5. Co-Investigators %%%%%%%%%%
%\CoI{first name}{last name}{institute}{email address}
\begin{investigators}
\CoI{Rachael}{Beaton}{Princeton University}{}
\CoI{Ana}{Bonaca}{Harvard University}{ana.bonaca@cfa.harvard.edu}
\CoI{Scott}{Carlsten}{Princeton University}{scottgc@princeton.edu}
\CoI{Masashi}{Chiba}{Tohoku University}{chiba@astr.tohoku.ac.jp}
\CoI{Andy}{Goulding}{Princeton University}{goulding@astro.princeton.edu	}
\CoI{Jenny}{Greene}{Princeton University}{jgreene@astro.princeton.edu}
\CoI{Masahiro}{Takada}{Kavli IPMU}{masahiro.takada@ipmu.jp}
\CoI{}{}{}{}
\CoI{}{}{}{}
\CoI{}{}{}{}
\CoI{}{}{}{}
\CoI{}{}{}{}
\CoI{}{}{}{}
%no need to fill \coiflag
\coiflag{}

%%%%%%%%%% 6. Thesis Work %%%%%%%%%%
%\thesis{student name}{thesis title}
\thesis{}{}

%%%%%%%%%% 7. Subaru Open Use Intensive Program %%%%%%%%%%
% Uncomment this line if this is an Open Use Intensive Program
%\intensive

\end{investigators}

%%%%%%%%%%%%
%% page 2 %%
%%%%%%%%%%%%

%%%%%%%%%% 8. List of Applicants' Related Publications %%%%%
\begin{publications}
    Price-Whelan, A.~M., \& Bonaca, A.\ 2018, ApJ, 863, L20.

    Bonaca, A., \& Hogg, D.~W.\ 2018, ArXiv e-prints, arXiv:1804.06854.

    Pearson, S., Price-Whelan, A.~M., \& Johnston, K.~V.\ 2017, Nature Astronomy, 1, 633.

    Price-Whelan, A.~M., Sesar, B., Johnston, K.~V., et al.\ 2016, ApJ, 824, 104.

    Price-Whelan, A.~M., Johnston, K.~V., Valluri, M., et al.\ 2016, MNRAS, 455, 1079.

    Price-Whelan, A.~M., Hogg, D.~W., Johnston, K.~V., et al.\ 2014, ApJ, 794, 4.
\end{publications}

%%%%%%%%%% 9. Condition of Closely-Related Past and Scheduled Observations %%%%%
%\relatobs{proposal ID}{title}{observational condition}{achievement(%)}
\begin{relationto}
\relatobs{}{}{}{}
\relatobs{}{}{}{}
\relatobs{}{}{}{}
\relatobs{}{}{}{}
\relatobs{}{}{}{}
\end{relationto}
% APW: not relevant!

%%%%%%%%%% 10. Post-Observation Status and Publications %%%%%%%%%%
%\pastrun{year/month}{proposal ID}{PI name}{status:completion/reduction/analysis}{status:publication}
\begin{previoususe}
\pastrun{}{}{}{}{}
\pastrun{}{}{}{}{}
\pastrun{}{}{}{}{}
\pastrun{}{}{}{}{}
\pastrun{}{}{}{}{}
\pastrun{}{}{}{}{}
\end{previoususe}
% APW: what do here?

%%%%%%%%%% 11. Experience %%%%%%%%%%
\experience{}
% APW: Please briefly describe your experience, ability, need of assistance, etc. for making observations with Subaru.


%%%%%%%%%%%%
%% page 3 %%
%%%%%%%%%%%%

%%%%%%%%%% 12. Observing Run %%%%%%%%%%
%\run{instrument}{# of hours}{moon phase}{moon distance}{seeing}{transparency}{airmass}
% "tothours" is calcurated automatically by ProMS. No need to fill \tothours{}.
\begin{observingrun}
\run{}{}{}{}{}{}{}
\run{}{}{}{}{}{}{}
\run{}{}{}{}{}{}{}
\tothours{0}
\minhours{0}
%\secondchoice{}
\orcomment{}
\end{observingrun}

%%%%%%%%%% 13. Instrument Requirements %%%%%%%%%%
%\begin{schedule}
\instruments{}
% Uncomment the following line if you want to make remote observations at Hilo.
%\remoteobs
% Uncomment the following line if you want to make remote observations at Mitaka.
%\remotemtk

%\end{schedule}

%%%%%%%%%% 14. List of Targets %%%%%%%%%%
%\target{name}{RA}{DEC}{magnitude}
% Equinox is J2000.0 unless otherwise specified in Comments on Targets
% Specify the coordinate in the format of hhmmss.ss +/-ddmmss.s for RA and Dec.
\begin{targets}
\target{}{}{}{}
\target{}{}{}{}
\target{}{}{}{}
\target{}{}{}{}
% no need to fill \targetflag
\targetflag{}
\targetcomment{}
\end{targets}


%%%%%%%%%%%%
%% page 4 %%
%%%%%%%%%%%%

%%%%%%%%%% 15. Observing Method and Technical Details %%%%%%%%%%
\begin{technicalinfo}
\end{technicalinfo}


%%%%%%%%%% 16. Public Data Archive of Subaru %%%%%
\begin{smoka}
% Uncomment the following line if you checked the public Subaru Telescope archive.
%\smokacheck
% If you propose observations in spite that there are available data in the public data archive,
% please denote the reason in the following field.
\smokacomment{}
\end{smoka}

%%%%%%%%%% 17. Justify Duplications with the HSC SSP %%%%%%%%%%
\hscssp{}


%%%%%%%%%%%%%
%% page 5+ %%
%%%%%%%%%%%%%

%%%%%%%%%% 14+. List of (more) Targets  %%%%%%%%%%
%*** (optional) ***
% If you have more targets, please uncomment the following
% 3 lines and enter them here.
%\begin{moretargets}
%\target{}{}{}{}
%\end{moretargets}




\end{document}
